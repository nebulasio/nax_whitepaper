\section{NAX分析}
\subsection{发行量分析}
我们给NAX设置了一个总的发行量上限为100亿(\(10^{10}\)),那么可以推导出第0期预发行的规模如下:
\begin{equation}
  \sum_{i,j} K_{i,j} = \sum_i C_0 \mu^i = \frac{C_0}{1-\mu}
\end{equation}
  令此上界为100亿(\(10^{10}\)),可解出\(C_0 = 10^{10}(1-\mu) = 1.0\times10^7\)。

\subsection{发放比例与质押率关系}
发放比例和质押率的关系,如图\ref{func}根据需求,发放比例的取值在0到1之间,例如,质押率为30\%时增发比例为70\%,质押率为50\%时增发比例为90\%。其中\(l=1.52\), \(m=-3.88\), \(n=3.36\)。

\subsection{权重与币龄的关系}
为了鼓励早期参与质押的用户,同时激励新用户加入质押,因此系统币龄产生的权重做了进一步处理。随着周期数的增加,当质押周期超过$90$后,权重将趋于平稳达到最峰值$1$,而新用户的起始权重从$0.67$开始。根据上述方程\ref{func}所示,可以计算出相应的系数,其中\(a=0.005\),\(b=-0.3\),\(c=0.2\),该函数效果如图\ref{weight}所示。

\subsection{追加质押对币龄的影响}
为了便于在原地址上追加质押,用户只需要对质押合约再进行一次契约交易。系统会根据过去质押和新质押的数量,重新计算出平均币龄。例如,用户当前质押$100$NAS,质押了$10$个周期,用户再追加质押$100$ NAS并通知合约后。那么新的质押数量为$200$ NAS,而币龄则是两次质押的平均值为$5$个周期。
