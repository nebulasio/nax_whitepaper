\section{NAX通证经济设计}
\subsection{设计目标}
公链通证经济设计的目标除了形成正向博弈、激活生态之外还应当符合一定的原则,例如:公平受益、正向激励、逻辑简洁,场景多样等。总而言之,公链的通证经济需要公正的获取场景和多样的消耗场景,拥有较高的持有价值,从而促进公链的发展和壮大,使得整个生态变得更加有活力和动力。总结起来NAX的设计目标是:

\begin{enumerate}[\hspace{2cm}(i)]
    \item 公平受益
    \item 正向激励
    \item 简洁有效
    \item 场景丰富
\end{enumerate}

\subsection{核心机制}

\subsubsection{资产公平性}
公链通证经济的有效性的根本来自于资产获得的公平性和正当性。资产的获取规则应当简单透明,对绝大多数人应当是信息对称的。在星云生态中,用户所持有的NAS资产具有相对的公平性和正当性,因此通过质押(Staking)NAS的方式作为获取通证权益的主要途径是符合要求的。不会因为规则不清晰、信息不对称,或者系统漏洞导致资产分配不公平的现象,并使得博弈的结果是有利于公链生态发展的方法。

\subsubsection{去中心化质押 - dStaking}
传统的中心化的质押方式是用户将资产转入到智能合约当中暂为保管,资产安全问题维系在一个智能合约中,合约资产安全性问题历史上屡见不鲜,攻击者利用了合约漏洞使得投资者产生巨大的经济损失,也使得投资者对合约安全性产生担忧。另一方面,质押同时也给公链项目方带来非常大的压力,大量资产保存在一个合约当中,使得合约的管理及安全性成为一个发展瓶颈。区块链核心是去中心化及资产的确权性,质押的目的是减少流动性,并不是获得资产的本身。

我们提出了一个新式的质押方式:去中心化质押(dStaking - decentralized Staking),如图\ref{fig:dStaking}所示,在确保用户资产仍然属于用户的前提下,用户与质押合约签定的质押契约记录在合约上。质押合约的作用只是通过链上随机检查用户是否仍在履行契约。当地址的余额大于等于契约的数额的时候,被认为是有效质押,否则将视为退出质押。dStaking的优势在于:
\begin{enumerate}[\hspace{2cm}(a)]
    \item 保障用户对资产确权
    \item 提高质押积极性
    \item 提高资产安全性
\end{enumerate}

\begin{figure}[htbp]
  \centering
  \includegraphics[width=1\textwidth]{../common/dStaking.pdf}
  \caption{中心化质押 vs 去中心化质押\label{fig:dStaking}}
\end{figure}

\subsubsection{NAX发行模型 - NDM}
在保障资产发行的公平性与正当性基础上,确保用户质押资产所有权神圣不可侵犯的前提下,用户贡献出资产的流动性以获取相应的生态权益,我们称这种新式发行模式为NDM(Nebulas Devotion Mining)。NAX的预计发行总量上限是100亿(\(10^{10}\)。 发行周期为每6,000高度(一天左右),每周期预发行数量呈递减规律,衰减系数$\mu=0.999$,预计12年左右发放完全。预发行NAX数量随周期数增长的变化图\ref{dist}所示,累积预发行NAX数量如图\ref{acc}所示。

\begin{figure}[h]
\centering
\begin{minipage}[5cm]{.45\textwidth}
\centering
    \begin{tikzpicture}[scale=0.8]
    \begin{axis}[
        axis lines = left,
        xlabel = {周期数},
        ylabel = {每周预发行NAX数量},
    ]
    \addplot [
        domain=-0:4000, 
        samples=1000,
        color=red,
    ]
    {1.0*10^7*0.999^x};

    \end{axis}
    \end{tikzpicture}
\caption{预发行数量与周期关系}\label{dist}

\vspace{\baselineskip}
\end{minipage}\qquad
\begin{minipage}[5cm]{.45\textwidth}
\centering
    \begin{tikzpicture}[scale=0.72]
    \begin{axis}[
        axis lines = left,
        xlabel = {周期数},
        ylabel = {累积预发行NAX数量},
    ]
    \addplot [
        domain=-0:4000,
        samples=1000, 
        color=red,
    ]
    {10^10*(1 - 0.999^(x+1))};

    \addplot [
        domain=-0:4000, 
        samples=1000,
        color=blue,
    ]
    {10^10};
    \end{axis}
    \end{tikzpicture}
\caption{累积预发行数量与周期关系}\label{acc}
\end{minipage}
\end{figure}

\subsubsection{动态分发策略}
动态分发策略是指,系统会根据一些变量决定实际分发NAX的数量,以促进生态的正向博弈。在NAX起初,我们引入质押率影响因子,根据质押率的增加或减少,动态调整实际分发比例。将来根据需要,我们还将引入更博弈的因素。如图\ref{fig:dynamic_dist}所示,周期$i$内系统预分发$C_i$ 个NAX给当前周期内正在质押的用户。在实际分发的过程中,根据当前的质押率水平$\lambda$(参与质押NAS的总量/总体NAS流通总量, 0<$\lambda$<1),实际分发NAX为:$C_0 \mu^i\lambda$。

\begin{figure}[h]
  \centering
  \includegraphics[width=0.8\textwidth]{../common/dynamic_dist.pdf}
  \caption{动态分发策略示意图 \label{fig:dynamic_dist}}
\end{figure}

\subsubsection{周期内分配策略}
在一个分发周期内,不同的质押周期数将获得不同的分发权重。系统根据每个质押用户的质押数量$V_{i,j}$以及质押权重\(f(T_{i,j})\)决定最终所获得的NAX分发量。假如第$i$周期内,有$N$个地址正在有效质押,其中第$j$地址质押数量为$V_{i,j}$, 有效质押周期数为$T_{i,j}$。因此,该地址所能分发到的NAX数量$K_{i,j}$如以下公式所示。

\begin{equation}
  K_{i,j} = \frac{V_{i,j} f(T_{i,j})}{\sum_j V_{i,j} f(T_{i,j})} \lambda_i C_i
\end{equation}

其中,\(f(T_{i,j})\)是第\(i\)期用户\(j\)的质押的有效权重函数。质押权重与质押周期数的关系,如下公式所式,其中函数关系如图\ref{weight}所示。
\begin{equation}
  f(T) = 1 - \frac{\sqrt{(aT+b)^2+c^2}-(aT+b)}{2}
\end{equation}

\begin{figure}[h]
\centering
    \begin{tikzpicture}[scale=0.75]
    \begin{axis}[
        axis lines = left,
        xlabel = {质押周期数},
        ylabel = {有效权重},
    ]
    \addplot [
        domain=0:365,
        samples=200,
        color=blue,
    ]
    {1-(sqrt((0.005*x-0.3)^2+0.2^2)-(0.005*x-0.3))/2};

    \addplot [
        domain=-0:365, 
        samples=200, 
        color=red,
    ]
    {1};
    \end{axis}
    \end{tikzpicture}
\caption{质押的有效权重与周期数的关系}\label{weight}
\end{figure}

公式中的参数$a$,$b$,$c$等将会在附录中讨论。总的来说,同一个周期内,系统会根据质押数量以及相应的质押时间长短来分配发行的总量,以达到公平的效果,即质押数量越多,质押时间越长,所分配到的发行数量也会更多。但同时使得让新进来的质押用户有更高的积极性,新用户的权重也会维系在一个可观的水平上。 该设计会达到以下博弈场景:

\begin{enumerate}[\hspace{1cm}(a)]
  \item 早期参与质押的用户,有更大的概率获得更多的系统发行
  \item 随着质押率增加,系统发行数量也会相应提高,以鼓励更多人加入质押
\end{enumerate}

\subsubsection{NAX生态基金池}
为了便于在生态上更好地在投资、孵化,扶持等活动中有更多的空间和主动权,将设立一个NAX生态基金池。系统在实际分发过程中, 将分发给质押用户的NAX中,扣除$5\%$转到这个基金池中,该基金池接受社区监督。具体的活动内容将在白皮书发布之后披露。

\subsection{合约框架}
NAX是扩展性的NRC20合约,由一组合约组成,并配合多签合约管理整个合约里的数据和参数,详细如图\ref{fig:nax_framework}所示。

\begin{figure}[h]
  \centering
  \includegraphics[width=0.8\textwidth]{../common/nax.pdf}
  \caption{NAX 合约组件示意图 \label{fig:nax_framework}}
\end{figure}
