\section{Present and Future Application Scenarios}
During the design of the NAX token model, we focused on creating a effective issuance model which respects fairness, legitimacy and authority of the asset. Combined with the ecological characteristics of Nebulas, we chose to distribute more incentives to application scenarios. For example, motivation and utilization scenarios in application scenarios can be varied, diverse, and unique.

In this chapter, we will look at some prospects for existing or upcoming scenarios within Nebulas. By revieing the scenarios, we can clearly demonstrate the tandem role of NAX within the Nebulas ecosystem.

\subsection{Ecosystem Contribution Incentive}
In the Nebulas white paper, the contribution-proof consensus algorithm Proof of Devotion (PoD) and the vision of Nebulas was shared: "Fair value for all via decentralized collaboration" and combined with the launch of the Go Nebulas platform in early 2019, Nebulas has constantly explored ecosystem contribution. These have all been important steps for Nebulas to move further towards the creation of the Autonomous Metanet. To this end, we put forward a unique pledge investment fund as a proof of equity incentive which can be applied to different scenarios.

The specific operation Pledging is to fund the project investment (usually in the form of NAS). The NAX equity of this part of the fund pledge will be used as the NAX incentive equity fund for ecosystem scenario.

\subsubsection{Go Nebulas Incentive}
The Nebulas Foundation will invest no less than $3 million in NAS to fund projects on the Go Nebulas platform and if needed, will be ready to invest additional funds on demand. This amount of money can also be invested in pledging and the resulting equity will be used to Incentivize the large and small contributions of users that are made on the Go Nebulas platform. For example, in addition to receiving NAS funds as a reward, they will also receive NAX incentives under the rules established by the Go Nebulas platform as an equity right contribution to the Nebulas ecosystem. The corresponding rights and governance can be exercised in the Nebulas ecosystem. Detailed incentives will be developed by Go Nebulas' operations team managers and community participants.

Incentives can be divided into the following categories:
\begin{enumerate}[\hspace{1cm}(a)]
	\item core infrastructure
	\item market expansion
	\item promotion
	\item creating proposals and participation
\end{enumerate}

In addition to being an important means of investing within community building and receiving NAX incentives, the Go Nebulas platform is an important use-case scenario for NAX. Utility scenarios include (but not limited to) the following categories:
\begin{enumerate}[\hspace{1cm}(a)]
	\item create and initiating a proposal
	\item passing and rejecting proposals
	\item pledge twords the progress of the project
\end{enumerate}

\subsubsection{Foundation Core Member Incentive}
Members of the Nebulas Foundation core team which includes part-time staff will receive NAX benefits from the Foundation's pledge as additional contribution in addition to receiving their appropriate salary.

\subsection{On-chain Governance Scenario}
Within the Nebulas ecosystem, there will be a variety of community selection and election activities. For each event, we will encourage greater participation and during events, different methods of NAX voting will be used such as destroying or returning utilized NAT which will motivate users for their participation.

For example a vote that will soon be proposed will be how to handle the 35 million NAS community reserve fund.The Nebulas Foundation has proposed to destroy this fund and we will ask the community to contribute options on the destruction (partially or fully). One possible solution is to launch a NAX utilized monthly vote to decide how much of the fund to destroyed every month. For example, if (\alpha\) \% , \(\alpha\) \% is the current NAS pledge rate in the share of liquidity. Of course, the community can also provide a more effective program and then jointly decide the processing of these assets.

\subsection{Node Campaign}
With the advancement of PoD development, the decentralization of Nebulas is aninevitable path. NAX will likely become a mechanism and credential for the Nebulas node campaign. It can effectively combine PoD technology innovation into the foundation and direction of the new consensus algorithm. The possible ways for a node to proceed are as follows:
\begin{enumerate}[\hspace{1cm}(a)]
	\item node candidates are selected via NAX vote.
	\item participants in the node campaign will need to destroy a corresponding amount of NAX and Pledge NAS.
	\item the community can crowdfund NAX to become a node and receive the benefits of being a node.
\end{enumerate}

\subsection{Ecosystem Advancement}
Developed by the Nebulas Foundation, the ecosystem products are an important element for the ecological use of the Nebulas blockchain. These products include the current and future Nebulas products incubated by the Foundation, such as NAS nano Pro, NAS nano, Web Explorer, Nebulas DEX, etc... As NextDAO advances and community governance progresses, there will be many more NRC20 tokens and governance projects within the community. These new assets have a strong demand these ecosystem tools. With limited resources and in order to have Nebulas ecosystem related products developed more improved projects, it is possible to use NAX as a platform to select excellent project tools and funding. These funds can also be used for activities and incentives for ecosystem projects. Funding is used to invest in the construction of the platform.
