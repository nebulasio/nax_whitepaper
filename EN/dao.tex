\section{NextDAO}
\subsection{Blockchain Collaboration}
With the emergence of the Ethereum ERC20 token a new financing method based on smart contracts via blockchain  led to a very low asset issuance expense and with the support of various blockchain trading platforms, tokens were able to obtain liquidity. This new capability led to early investors being able to exit their investment at any time leading to a sinifigently reduced investment length. This new paradigm did not however solve post-financing issues for projects and as a result, this led to the creation of many deceitful projects.

Blockchain technology is essentially a decentralized, trustless, game-based autonomous system. Its real charm is its open collaboration model based on consensus mechanisms under the idea of decentralization. The most famous attempt at blockchain collaboration is the DAO (Decentralized Autonomous Organization) project on Ethereum. The DAO ideology is derived from equity financing technology on the Ethereum blockchain which creates a decentralized management structure by utilizing code for organizational rules and the decision-making process. This capability eliminates the need for written documentation and reduces the requirement for human intervention. In 2016, Ethereum's DAO project was hacked and tens of millions of dollars worth of ETHER was stolen which eventually led to a hard fork on the Ethereum blockchain. Although The DAO was ultimately not successful, it was a great attempt and a lot of lessons were learned which has assisted follow-up projects.

We believe that collaboration via blockchain still has the following issues:
\begin{itemize}
	\item \textbf{Collaboration role diversification}
	In the early Bitcoin community, the community consisted of miners and coin holders. With the emergence of Ethereum, more and more people were exposed to the blockchain and created additional user groups such as developers and application users. Due to this, the distribution of rights and responsibilities of different user roles was challenged.
	\item \textbf{Single incentive mode}
	At present, most of the public chain consensus incentives are focused on mining incentives based on PoW or PoS. With more user roles within a blockchain ecosystem, a single incentive model cannot adequetly give incentives to all participants.
	\item \textbf{Ecosystem models lack fairness}
	The current blockchain ecosystem model is not fair to most participants leading to some mainpuating the system for their personal benefit.
\end{itemize}

In order to solve the problems of blockchain collaboration listed above but not limited to, Nebulas proposes a financial blockchain collaboration framework entitled "NextDAO." This platform will include public chain collaboration, governance and decentralized finance (DeFi).

\subsection{Public-Chain Token Economy}
The Token Economy is embodied in an economic model that includes the generation, circulation, repurchase and incentives of issued tokens. In the conventional economy, the manifestation of the asset include: currency, notes, points, stocks, claims, usage rights, ownership, and so on. The generation, circulation, repurchase and incentives of these rights are all guaranteed by centralized institutions. In the world of blockchain, a decentralized economic model has emerged. With this emergence, a typical use-case of a public chain token economy is the Ethereum ERC20 token. It greatly facilitated the speed of financing and distribution leading to a stimulated economic prosperity on Ethereum as well as driving the development of the entire blockchain industry. 

Therefore, the value and innovation of the public chain is not only due to the innovation of this technology itself but also the model and commercial innovation brought by technology.

Building a Token Economy for a public chain is just as important as developing the underlining public chain technology. The biggest problem facing public chain incentive is human nature which eventually becomes a human-to-human game. As a result, the participants are aimed at obtaining their best interests rather than completing the best ecological construction. 

Most public chain projects fail in comparison when trying to compare the power of the Ethereum community. With this in mind, it's critical to design a token economy that suits it's ecosystem. A designed token economy must align with the expansion of consensus, the development of the community and a model that presents a positive ecosystem to all which will bring long-term benefits to the system. The positive development of user incentive is needed in order to promote the development of blockchain technology and its commercial expansion.

The public chain can be thought of as a public resource platform where any user can utilize or trade on. Therefore, the public chain does not belong to any individual and is a public resource. In order to avoid the tragedy of the commons ~\cite{TragedyOfTheCommons}, an effective and high-liquidity economy is needed to form a long-term effective positive ecosystem which also consists of a excelent governance environment, expansion of community consensus and as a result will lead to better technological innovation and development.

Nebulas will actively develop a unique token Economy of its own and will continue to build a better collaboration platform where every participant will benefit fairly.

\subsection{Nebulas Token Economy - NAX}
As a public chain, Nebulas will establish an ecosystem framework based on its own strengths and characteristics. We aim to build a system of fairness, validation, collaborative innovation, incentive contribution, stimulating positive ecosystem development, strong community consensus and developing unique public chain technology. The core logic of NAX and application scenarios are described in detail in the following sections.
