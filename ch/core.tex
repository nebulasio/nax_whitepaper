% !TEX root = main.tex
\section{核心系统设计}

\subsection{质押NAS返NAX}
标准:
\begin{enumerate}
\item 四个周期(发行量)调整一次
\item 质押率 == 算力
\item 算力会影响难度
\item 衰减速率
\end{enumerate}

示例:
\begin{enumerate}[a.]

\item 一年减半(2,500,000), 周期:100,000 (两周多)衰减一次\\
衰减系数:0.973\\
math.pow(0.973, 25) = 50\%

\item 一年减半(2,500,000), 周期:50,000 (一周多)衰减一次\\
衰减系数:0.986\\
math.pow(0.986, 50) = 49.4\%

\item 两年减半(5,000,000), 周期:100,000 (两周多)衰减一次\\
衰减系数:0.986\\
math.pow(0.986, 50) = 49.4\%

\item 两年减半(5,000,000), 周期:50,000 (一周多)衰减一次\\
衰减系数:0.993\\
math.pow(0.993, 100) = 49.5\%

\end{enumerate}

新提议(质押消耗NAS) -- 感觉不是特别友好

如果我们需要维护一个质押率,达到一定的博弈平衡,可以添加,根据质押数量,消耗NAS的质押场景,也就是说,当质押分配到的NAX不合算的时候,用户可能会取消质押。当质押数变小的话,又会使得质押所得的NAX变得更多,所以又有人开始质押。

质押消耗的NAS可以被看作是BTC挖矿中消耗的电费。

收集的消耗的NAS会收集成为社区建设基金:Go Nebulas

公式如下:其中

        第\(i\)期用户\(j\)获得的NAX: \(K_{i,j}\)

        第\(i\)期用户\(j\)的质押量: \(P_{i,j}\)

        第\(i\)期用户\(j\)的质押时间: \(T_{i,j}\)

        第\(i\)期初始总增发量:\(C_i\)

        第\(i\)期增发比例: \(\lambda_i\)

        衰减系数:\(B\)

        h是质押高度总和,v是质押数量

\begin{equation}
  K_{i,j} = \frac{P_{i,j} T_{i,j}}{\sum_j P_{i,j} T_{i,j}} \lambda_i C_i
\end{equation}
\begin{equation}
  \lambda_i = f(\sum_j P_{i,j} T_{i,j})
\end{equation}
\begin{equation}
  C_i = C_0 B^i
\end{equation}

\subsection{质押返率\(\lambda\)}
\begin{enumerate}[a.]
  \item 时间。难度问题。早期质押, \(\lambda\)更高,正向量
  \item 预期?质押高度为质数, \(\lambda\)更高 x
  \item 运营活动。。。返回会越来越多?上一高度交易量越高, 此高度 \(\lambda\)更高 x
  \item 质押率(算力)越高, \(\lambda\)越高,质押率下降,难度下降,收益上升。质押率上升,难度上升。
\end{enumerate}

\begin{equation}
\lambda = (f(a, c, d) + g(h) b) / h
\end{equation}

\subsection{增发周期设定}
可以每100000高度分发一次,也就是高度满足以下性质:

H = h mod 100000 == 0

\subsection{取回质押策略}
满足性质:v越高,取回周期t越长,滞后返还, B是基本量级

B(v) =  floor(sqrt(v) - B) * t

\subsection{系统手续费}
每次增发的时候,新增发所得的4\% 转入NAX的专属项目基金。

矿工费?

固定手续费。发行费。铸币税。

项目团队的预期收益。

