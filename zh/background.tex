\section{Token Economy}
Token Economy,即“通证经济”,具体表现为包含通证产生、流通、回购、激励的经济模型。在现实生活中,通证表现形式包括:货币,票据,积分,股票,债权,使用权,所有权等等。这些权益的产生、流通、回购、激励都依靠中心化的机构来保证。在区块链的世界里,相应的去中心化经济模型也应运而生,大致可以分为两大类:

\begin{itemize}

	\item \textbf{去中心化金融(DeFi)}

	去中心化金融 (Decentralized Finance),运用区块链技术来解决传统金融的问题:金融体制不平等、审查流程繁琐、缺乏透明性等。包括:稳定币、借贷、支付、衍生品、去中心化交易所等。

	\item \textbf{生态经济系统}

	在一个公链生态或者一个区块链应用中,所建立的一套完整、自恰的经济模型。

\end{itemize}

\subsection{公链的Token Economy}

公链生态的Token Economy中的典型的案例是以太坊的ERC20。使得以太坊成为全球区块链融资的平台。由于大大便利了融资与分配的速度,刺激以太坊的生态繁荣,也同时带动了整个区块链行业的大发展。因此,公链的价值和创新不仅仅源于在“不可能的三角”上的技术本身的创新,也来源于技术所带来的模式和商业创新。

建立一个适合公链的Token Economy与发展公链技术本身同样重要。传统的Token激励模式,大多有漏洞,或者绕不开羊毛党。公链激励面临的最大问题是人性,最终变成了人与人的博弈,即参与者以获取最大利益为目的,而不是以完成最好生态建设为目的。大多数公链远达不到以太坊的社区力量,因此建立适合自己的Token Economy变得尤其关键,关系到共识的扩大,社区的发展,一个正向的博弈的经济模型会给系统带来长远的正向发展刺激,这样才能带动区块链技术的发展以及寻求区块链的商业落地。

\subsection{质押经济 - Staking Economy}
公链发展Token Economy的一个主要尝试是通过Staking或者叫质押,Staking原自于基于PoS[x]共识,通过质押token获得挖矿权。随时时间的发展,Staking已经不仅权限是PoS Staking获得挖矿权益,现在也泛指通过质押获得额外的权益,票权甚至是新的token的增发。

\subsection{公链治理范式}
公链可以看作是一个公共资源平台,任何用户都可以在公链上交易。所以公链不属于任何个人,是一个公共资源。为了避免公地悲剧,需要有效Token Economy才能形成长期有效的正向博弈,拥有良好的治理环境,社区共识的扩大,从而才会更好的技术创新和发展。星云将会大力发展适合自己的Token Economy,坚持成为更好的协作平台,让每个参与者公平受益。
