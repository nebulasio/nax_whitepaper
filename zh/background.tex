\section{背景介绍}

其中,“NAT 链上投票”是核心手段,包括两部分:
1. 投票唯一介质:NAT 及其底层算法是星云链上治理的重要工具;
2. 链上投票流程。
本章节将对其进行重点介绍。
12
5.2 投票基本原则
星云生态通过星云主网实现链上投票。社区投出的每一票,将在星云链的区块上
公开透明地展现。在星云链的系统中,投票将遵循以下基本原则:
1. 投票的最基本单位是一个星云主网地址。
2. 星云链的投票权重将参照地址的星云指数。
3. 对系统有积极贡献的行为应被奖励更多的投票权,在投票的场景中,我们认为
投票行为是对星云系统有积极贡献的行为,应被激励更多的投票权。
5.3 投票方式
投票将通过星云主网上的投票智能合约实现,每个地址可以选择投赞同、反对、
弃权三种票。亦可不参与投票。
5.4 投票唯一介质:NAT
5.4.1 概述
• 名称: Nebulas Autonomous Token
• 代号: NAT
• 形式: NRC20 token
NAT 是星云指数的资产形态,它以NRC20 Token 的形式体现,并作为星云生态
治理场景中唯一的投票介质。


作为社区治理币,需要明确几个设计目标,最终NAX设计出来需要达到哪些目标。在确定一个系列目标的基础上,再进行机制的设计,所有机制的设计也将需要符合这个总的目标。为了使得机制设计简洁、清晰,一切与目标不相关的设计,都不应该随意纳入机制当中,以妨止最终偏离初衷。以下是几点关于治理币几个总的目标:

\begin{enumerate}[a.]
        \item 用于星云社区治理
        \item 星云社区贡献有效凭证
        \item NAX需要是符合通缩模型
        \item NAX有升值空间
        \item 销毁与增发达到一定程度的平衡
\end{enumerate}

星云治理的目的是实现星云愿景,聚焦去中心化协作。在谈论具体的星云治理措施之前,需要先了解当前新场景下出现的协作困境,以及星云的治理方式想要解决的问题。

\label{background}

人类是具有社会性的,我们对协作(Collaboration)并不陌生。即便是孤岛上的鲁滨逊,也和星期五磨合出了一套相处模式~\cite{robinson}。协作方式本身没有绝对高低优劣之分,在不同场景下,应选择适合该场景的一种或多种协作方式。随着科技的发展,协作场景已经从人与人面对面合作,升级到全球化的跨越多地区、并有多组织参与的协作。协作的目标结果也更为多变,成果从实物到虚拟,协作的时间跨度也变得更长、更灵活。

星云并不力求颠覆其它协作方式,也不排斥多种协作方式混用。星云尝试寻找更合适的协作方式,补足当前其它协作方式在新场景下的不足。新的应用场景有如下特点:

\begin{itemize}
	\item \textbf{信息交互从单⼀、简单的功能向复杂、多样发展。}
	
	去中心化的电子加密货币比特币的诞生让区块链可以记录交易信息。以以太坊为代表的第二代区块链系统提出了具有图灵完备性的智能合约,区块链变得可编程。逐步出现了链内、链外、“跨链”等不同场景的数据和资产交互问题。

	\item \textbf{用户角色也随之越来越多。}

	早期比特币社区只有矿工和持币者,有了以太坊之后出现了开发者、应用使用者等,越来越多的人接触到区块链,不同用户角色的责权利如何分配受到挑战。

\end{itemize}

当前常见的协作方式在新场景下的不足举例:

\begin{enumerate}
	\item 

	\textbf{中心化治理无法应对复杂的新场景。}

	区块链技术本质上是⼀种去中心化、⾮信任、基于博弈的⾃治体系,其真正的魅力是在去中⼼化思想下,基于共识机制的开放协作模式~\cite{whitepaper}。但有的区块链项目干脆反其道而行之,使用中心化方式进行治理,如直接通过核心仲裁法庭直接对黑客进行判决。这种方式的正当性和公平性难以得到保障。面对复杂的数据交互形态和丰富的用户角色,中心化的单一评判标准很难做到广而全面,引起社区成员反抗。2019年1月11日,EOS Authority发起了关于是否废除ECAF(EOS核心仲裁法庭)的投票,支持废除的比例高达98\%~\cite{DeleteECAF}。

	\item 

	\textbf{现有的去中心化治理方式规则不统一。}

	例如比特币社区,对矿工和持币者等不同角色的用户来说,对应的规则并不相同,且不明确。此类去中心化治理方式容易引起社区发展目标不明确,难以有效组织和迭代升级。

	\item 

	\textbf{传统去中心化协作方式常现公地悲剧~\cite{TragedyOfTheCommons}。}

	传统去中心化协作项目,如大量开源社区,本身利益模型不明确,经费来源往往主要靠捐赠,升级进化过于依赖开发者的兴趣,经常会出现公地悲剧、生态进化缓慢等问题。使用公共资源(如开源代码)的人多,作出贡献的人少。依托大公司和企业定向捐赠的开源项目又经常会被大公司的发展方向牵制,成为企业的附属。

	区块链技术因为有了代币的存在,让我们有机会通过提供可持续的激励,构建良性经济体来解决去中心化协作的基本困境。

	\item

	\textbf{早期区块链项目共识机制激励不全面,社区参与度低。}

	如比特币使用的共识机制工作量证明(Proof of Work,PoW~\cite{pow})仅专注挖矿激励,此种单一激励不能应对用户角色的逐渐丰富。去中心化的以太坊由于升级缓慢而饱受诟病。以太坊的升级提案需要得到社区广泛认同后由矿工完成操作之后才可执行,但整个生态中各方意见难以统一,以太坊的激励没有覆盖整个生态的不同用户角色。“事不关己”的现象导致以太坊升级提案参与度低,迟迟难以执行,生态发展受阻。

\end{enumerate}

目前不存在完善的方案可以解决上述问题。我们意识到,应对具备前所未有复杂度的新世界,新技术的诞生众望所归。

