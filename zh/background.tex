\section{Token Economy}
去中心化金融 (Decentralized Finance),简称 DeFi,也称为开放式金融,是近两年来区块链生态圈最热门的领域之一,尤其是,当下活跃在以太坊网络上的 DeFi 应用。

DeFi 试图用区块链技术,来解决传统 / 中心化金融存在的天然短板,诸如:金融体制不平等、审查流程繁琐、缺乏透明性、和潜在的交易风险(不履行合同义务的风险即坏账)等。

在 DeFi 系统中,用户在获取金融服务时,对个人资产和数据有独立控制权,而整个 DeFi 平台开放性很高,用户身处世界各地都可以随时参与。和传统金融模式相比,DeFi 不仅增加了透明性,而且整个金融协议通过智能合约自动执行,很大程度上减少了传统金融服务过程中潜在的各类风险,包括:人工干预成本、坏账率等。

DeFi 行业可以进一步细分为稳定币、借贷、支付、衍生品、去中心化交易所和资产等。本报告前半部分会概览 DeFi 行业现状,然后着重就稳定币、借贷与去中心化交易所三个类别做详细的数据分析和探讨。
当去中心化金融(DeFi)成为加密货币市场中最热门的关键词之后,关于「DeFi」是否真的真的未来趋势,还是昙花一现,成为近期讨论的焦点。

让我们用数字说话。DAppTotal 分析了 2019 年上半年 DeFi 行业的发展状况,并重点就稳定币、借贷市场、和去中心化交易所三大市场展开了系统全面的数据解读和分析,并形成本报告,希望给众说纷纭的市场提供清晰、可信的事实支持

整体而言,DAppTotal 数据统计显示,截至 2019 年 06 月 30 日,DeFi 行业的总锁仓价值为 14.9 亿美元,较之 01 月 01 日的 3.02 亿美元,半年时间增长了近 5 倍。

其他几个值得关注的发现还包括:

稳定币:目前大部分稳定币在以太坊网络上发行,价值和美元挂钩。在 2019 年上半年,稳定币交易量一直呈稳步增长之势,04 月份以来,伴随着加密货币价格的回升,稳定币市场进一步加速增长。据 DAppTotal 统计,仅 06 月 30 日单日,稳定币的链上总交易量就达到了 8.59 亿美元,这意味着,总流通量 50 亿美元的稳定币市场,平均每 5、6 天就会买卖换手一次。
去中心化借贷:该市场还处于早期发展阶段,仅有两年的历史,但其发展速度却很快。据 DAppTotal 统计,截至目前,借贷市场贷出最多的加密资产分别为:DAI 和 WETH,六月份的借贷总额 (贷出 +借入) 是 5.44 亿美元,较一月份的 3,400 万美元,增长了 16 倍。
去中心化交易所(DEX):和中心化交易所相比,去中心化交易所的整体市场规模还比较小。其主要优势价值在于透明性和安全性,而不足之处在于发展速度和市场流通性。目前 DEX 上流通的主要是中心化交易所未上线的币种。据 DAppTotal 统计,过去半年内 DEX 的月总交易额保持稳步增长,1 月份为 7,100 万美元,六月份增长至 2.88 亿美元。


\subsection{公链的Token Economy}

公链的价值来源必须来源于商业。而在公链上开展商业与项目的根本优势是天然的token激励模式。

但是目前的token激励模式有漏洞。这也是我认为的困局之一。

所有的尝试用token做激励的项目,都绕不过羊毛党的问题,因为金钱是最短路径激励。这最终变成了人与人的博弈,机制设置者与羊毛党之间的博弈,那么从这个角度去观察,本质上来说,羊党是所有token激励项目绕不过去的命题。

公链激励面临的最大问题是人性,即参与者以获取最大利益为目的,而不是以完成最好生态建设为目的。这个点我是相对悲观的。

如果今天我们是以太坊,可能不太需要去考虑这些。因为你做得足够早,社区力量还非常强悍。但作为一个后来者,我们需要去想好自己真正要做的事情是什么,经济模型设计要更合理 **。经济模型不仅是一个让生态平衡的力量,经济模型还要让这个生态有成长的空间。** 所以我们先给自己抛出一个框架,然后根据这个框架试着往前走。

我们有一些指导教授帮助我们一起规划经济模型的设计,他们很多是金融和财政背景出身。所以借鉴得比较多的框架有两个,一个是国家收税,所谓的财政政策;还有一个就是在金融系统里的行为,包括一个资产的价值,它的流动性是怎样的。

总结来说,主要就是宏观的经济,包括货币、财政,你可以想象央行和财政部把所有的行为都规范了,他们设计的模型能自动去调节所有经济和人的行为。调整的方式其实就那么几种手段,要么给予激励或者给予补偿,要么用税收加以控制。宏观经济和和金融系统里的一些行为研究对我们都很有大的帮助。

以前的经济模型设计更多都是和共识紧密结合在一起的,它可能主要就是为了共识服务的,比如怎么样用经济模型让矿工们达成共识,更多从这个方面去思考。所以产生的一些设计是加交易手续费,让矿工有动力去记账,POS 需要 Staking 的人保持在线,如果不这样就砍 Staking 的币等等。以前的经济模型更多是从共识形成的本身出发,用经济模型去推动共识的形成。

当然,所有的公链都是要达成共识。我们也是这样。这个没问题。但达成共识不是经济模型的全部。一个公链的目标到底是什么,公链想成为什么样的东西,有什么独特的价值必须存在?这些是经济模型需要完成的更宏大的目标。


我们会更多思考这些问题,而不是仅仅让公链达成瞬间的共识。这些可能是我们之前的团队很少想的,他们只是用奖惩的方式来保证共识不断进行下去,但是你这一秒可能系统状态是平衡的,但长期呢?有竞争的情况下呢?出现分叉呢?这些需要经济模型的设计者考虑周全,因为我们想要的是一个真的可以运行五十年一百年的系统。

公链可以看作是由矿工社区提供的一个去中心化开放平台,任何用户都可以在公链上交易。对于用户来说,这个公链不属于任何人,是一个公共资源。矿工则自己提供的服务期待回报,如果不能很好的补偿矿工,就会导致公共资源的消耗。这是个去中心化的系统,你不去做设计的话,每个人又必然会按照自己利益最大化的方式来损害公链的资源。

\subsection{质押经济 - Staking Economy}
如果我们看一看现在的 staking 经济格局,根据 Staking Rewards 网站的数据,截止至 4 月 30 日,PoS 项目的总市值大约是 175 亿美金,总 staked 比率是 30%,总 Staked 金额是 55 亿美金左右,乘上平均收益率 10%,现有产生的 Staking rewards 就有 5.5 亿元美金的市场,而且这个市场正在快速成长,2019 年明星项目,Cardano、Algorand、Polkadot、Difinity 都属于 PoS 共识项目,可预见未来以太坊由 PoW 到 PoS 的转型将是 PoS 生态的很大催化剂,目前 PoS 共识占整个加密货币市值约为 10%,今年可能在 10%-20% 之间,未来更长一段时间可能占据数字货币的半壁江山。

1.什么是Staking呢。

Staking 是一个动词,而 stake 的名词意思是「股权」意思,动词的 staking 相比之下就相当难以直接翻译,最贴近的中文翻译可以理解为「质押」。

Staking 是在 PoS 项目中一个特有的动作,通过对代币进行 staking (质押)的动作,持币用户可以获得 staking 收益或是 staking 的权益。

2.Staking的核心要义;【主动参与】。不进行这个动作,对持币人来说,币就是币,没有启动 staking 的权益功能。

Staking是鼓励代币持有者主动参与到区块链网络中的一个设计,把「主动参与」当作是一个投资的策略。

质押经济本质上来说也是一种挖矿,但和我们通常所说的比特币挖矿,以太坊挖矿不同。

比特币,莱特币,以太坊,BCH等这些数字货币都是基于工作量证明(POW)的数字货币,因此在这种机制下,产生新的货币都是比拼算力,所以就有了各种矿机。当下最流行,市场占有率最高的就是比特大陆的矿机。

当我们要参与这些数字货币的挖矿时,我们通常都是去市场买矿机,然后自己找机房或者将矿机托管给大矿场代运营。矿机每天挖到的币除去电费和运营费剩下的就是纯收入。

“Staking”(质押)则是另外一种挖矿方式。通常基于权益证明(POS)和代理权益证明(DPOS)的数字货币就采用的是这种挖矿方式。

在这种挖矿方式中,区块链系统中的节点不需要太高的算力,而只需要质押一定数量的代币,运行一段时间后就可以产生新的货币,而产生的新货币就是通过质押得到的收益。

\subsection{公链治理新范式}
