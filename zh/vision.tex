\section{星云理念概述}

\subsection{理念与愿景}
区块链技术本身不是一个全新的技术创新,而是作为一系列技术的组合(包括密码学,分布式系统,博弈论等)而产生的模式创新。比特币~\cite{Nakamoto2008}创造了“⼀个去中⼼化电⼦现⾦系统”的完美设计,从而打开了区块链世界的大门。以太坊~\cite{buterin2013ethereum}进一步提出了具有图灵完备性的代码的智能合约区块链框架,并发明了ERC20的范式,使得在区块链上融资变得更加便利。区块链技术也因此取得了空前的繁荣和发展。

在星云的白皮书~\cite{whitepaper}中也提出了自己的区块链理念,并持致力于践行“让每个人从去中心化的协作中公平受益”的愿景。同时也提出了自己的路径,其中包括:星云指数(NR), 开发者激励协议, 星云原力(NF)以及星云贡献证明(PoD)等。在过去两年中,星云链根据自身优势以及在区块链世界摸索前行的中总结的经验,将会有更多的探索和尝试。


\subsection{区块链协作}
随着科技的发展,协作场景已经从人与人面对面合作变得更灵活、更自由。区块链技术本质上是一个去中心化、非信任、基于博弈的自治体系,其真正的魅力是在去中心化思想下基于共识机制的开放协作模式。目前区块链的协作仍然存在着以下几个问题:

\begin{itemize}

	\item \textbf{协作角色多样化}

	早期比特币社区只有矿工和持币者,有了以太坊之后出现了开发者、应用使用者等,越来越多的人接触到区块链,不同用户角色的责权利如何分配受到挑战。

	\item \textbf{激励方式单一}

	目前大多数公链的共识激励还是以PoW, PoS为主的专注于挖矿的激励,事实说明单一激励不能应对用户角色的逐渐丰富。

	\item \textbf{公平与正向博弈缺失}

	为社区做出贡献的角色并没有得到对应的激励,使得整个区块链没有呈现出正向博弈的。

\end{itemize}