\section{星云理念概述}

\subsection{理念和使命}
2008年10⽉31⽇,中本聪(Satoshi Nakamoto)发布⽐特币⽩⽪书~\cite{whitepaper},从此我们迎来了⼀个有区块链的世界。经过十年的发展,⽐特币践⾏了其作为“⼀个去中⼼化电⼦现⾦系统”的初衷。以太坊[x]更进一步,为区块链世界提供了一个运行具有图灵完备性的代码的能用区块链框架。区块链技术在此之后也取得了空前的发展和繁荣。区块链技术本身不是一个全新的技术创新,而是作为一系列技术的组合(包括密码学,分布式系统,博弈论等)而产生的模式创新。在星云的白皮书[x]中提出了自己的主张和解决方案,并始终坚持致力于践行“让每个人从去中心化的协作中公平受益”作为星云使命,落地场景上以实现The Better DAO[x]为目标。


\subsection{区块链协作}
随着科技的发展,协作场景已经从人与人面对面合作变得更灵活、更自由。区块链技术本质上是一个去中心化、非信任、基于博弈的自治体系,其真正的魅力是在去中心化思想下基于共识机制的开放协作模式。目前区块链的协作仍然存在着以下几个问题:

\begin{itemize}

	\item \textbf{协作角色多样化}

	早期比特币社区只有矿工和持币者,有了以太坊之后出现了开发者、应用使用者等,越来越多的人接触到区块链,不同用户角色的责权利如何分配受到挑战。

	\item \textbf{激励方式单一}

	目前大多数公链的共识激励还是以PoW, PoS为主的专注于挖矿的激励,事实说明单一激励不能应对用户角色的逐渐丰富。

	\item \textbf{公平与正向博弈缺失}

	为社区做出贡献的角色并没有得到对应的激励,使得整个区块链没有呈现出正向博弈的。

\end{itemize}

\subsection{技术愿景}
在践行这个使命和达成目标过程中,星云提出了一些路径,其中包括:星云指数(NR)[x], 开发者激励协议[x], 星云原力(NF)以及星云贡献证明(PoD)等。在过去两年中,由于区块链技术得到了前所未有的发展,商业落地的场景和尝试也层出不穷。星云链在坚持最初的主张和愿景的同时,根据自身优势以及在区块链世界摸索前行的中总结的经验,将会有更多的探索和尝试。
