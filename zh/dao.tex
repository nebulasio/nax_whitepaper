\section{DAO}

\subsection{什么是DAO}

\subsection{什么是Better DAO}
2009年kickstarter的出现,为创意产品提供了新的互联网众筹的融资模式,但是kickstarter的众筹是公益性的捐赠,没有分红权,因此能够筹集到的资金有限,2018年Kickstarter上众筹金额最多的项目是520万美元,第20名是130万美元。而在Kickstarter基础发展起来的股权众筹则面临着政府的严格监管,同时股权缺乏退出渠道,没有流动性,投资参与人数少,因此很难成为创意产品的融资方式。

比特币、以太坊等区块链平台出现后,ICO成为一种新的融资方式,以区块链智能合约技术为基础,资产发行的成本很低,在各种区块链代币交易平台支持下,代币发行后即可上市交易获得流动性,早期投资人退出时间大大缩短,2017年,ICO融资达到高潮,我们应该看到,ICO成功的关键因素是其超高的流动性。但由于ICO没有任何监管,因此伴随着很多ICO骗局,给投资人造成损失,并最终ICO被主流国家禁止,在少数国家允许在监管机构注册许可后发行ICO,受监管后的ICO仍然没有降低融资成本。

为了避免ICO融资形成的缺点,寻找更好的ICO机制,许多人进行了探索,2017年Vitalik对DAO和ICO结合的形式——DAICO进行了探讨,2017年,Simon de la Rouviere第一次提出了以债券曲线作为一种代币发行机制的概念,而bancor项目则提出了以bancor协议作为代币发行机制,以bnt代币为储备金。2018年,Wilson Lau讨论了以双曲线债券曲线模型为项目方融资模式。

理论上可以作为任何组织的股权融资管理平台,但是我们认为最合适的是有清晰盈利模型和盈利预期的轻资产的文化创意创业公司,例如游戏、动漫等领域。我们可以通过DAO的组织形式来组织产品的生产,而产品本身不一定必须是应用了区块链技术,以游戏为例,游戏本身可以不需要是区块链游戏,可以是面向传统游戏玩家的作品,从而增加盈利预期。未来,当支付本身也发生在区块链上后,我们可以形成链上的融资、支付、分红的完整去信任化流程,投资人的权益得到最大化保障,从而降低投资成本,促进投资市场的繁荣。


\subsection{A Better DAO - NAX}
引入星云NAX