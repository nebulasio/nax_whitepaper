\section{NAX 应用场景展望}

各使用场景可以自行设定,奖励,销毁策略。

\subsection{Go Nebulas平台}

基金会资助Go Nebulas平台的300w NAS, 以及今后系统增发的部分NAS作为资助Go Nebulas平台经费,这些资金也将同时作为质押以获得NAX,作为社区贡献的附加奖励。

\subsubsection{分发场景}

星云社区化项目制的开放性平台,社区贡献者可以由分为几个角色:社区开发者,社区推广者。社区贡献者通过Go Nebulas上做项目,将会获得NAS的工资奖励, 同时将会获得额外的NAX的社区贡献奖励/凭证,根据项目的优先级,定义不同比例的NAX返比。

社区贡献者分类

\begin{enumerate}[a.]
  \item 项目开发
  \item 运营/PR
  \item 市场拓展
  \item 拉新
  \item 基础设施:
  \item 主网
\begin{enumerate}[i.]
  \item NAS nano Pro
  \item 硬件钱包
  \item 新钱包对接
  \item 跨链合作
\end{enumerate}
\end{enumerate}

为了鼓励社区贡献者的积极性,我们初步确定以下的返NAX比例参数:

普通项目 :S * x * N (NAS reward)

优先项目(基础设施):T * x * N (NAS reward)

(s < t)

\subsubsection{消耗场景}
\begin{enumerate}
\item 创建提案(消耗相应的NAX)
\item 开发提案(开发时会销毁等比的NAX, 提前完成的工作,节省销毁的数量)
\item 投票通过提案和结果,需要销毁后NAX(通过或以获得120\%返还,失败可以获得110\%的返还)
\end{enumerate}

推进步骤
\begin{enumerate}
\item 补发过去参与过贡献的人的奖励?
\item 以后获得项目资金的人将获得相应NAX奖励
\item 增加GN邀请奖励,受邀请的贡献者获得的NAX后,邀请人会获得额外10\%的NAX
\end{enumerate}

\subsection{星云治理 - 提升社区参与度}
为了鼓励社区使用NAX作为社区治理的工具,以及鼓励大家社区治理的参与度,我们将设置一些投票场景,部分场景投票将做会有返还奖励。
投票的场景

\subsubsection{节点竞选 -- PoD}
\begin{enumerate}[a.]
  \item 从节点候选人里可以选出节点
  \item 投票销毁后会返还(若当选,获得120\%返还,失败有110\%返还)
  \item 参与节点竞选, 需要销毁1000W NAX,具体参与需要再调整
  \item 成为节点, 被选中成为出块节点后,参与到节点出块,需要质押NAS(无NAX返还)
\end{enumerate}

\subsubsection{理事竞选}
\begin{enumerate}[a.]
  \item 从候选人里选出相应的理事
  \item 投票销毁后会返还(若当选,获得120\%返还,失败有110\%返还)
\end{enumerate}


\subsection{星云生态上币费}
NAS nano Pro \& Explorer \& DEX等各类生态平台的上币费/手续费
随着Next DAO的推进以及社区治理的前进,社区里将会出现越来越多的Token和治理尝试。这些币种都将需要相应的工具支持,所需要上NAS nano Pro和Explorer的需求。资源空间有限的情况下,我们可以采取增加上币费的需求,比如需要在NAS nano Pro和Explorer上币的NRC20需要缴纳 500w NAX(参数可调)的上币费。其中20\%归集给NAS nano Pro和Explorer管理团队 (开发者和运营者)80\%会被销毁

\subsection{社区预留NAS销毁计划}
与其将3500w NAS一次投票销毁,其实可以把销毁做成一项长期的社区投票活动,由社区来决定这个事情的发生。

销毁细节:

每个自然月1号发起一次投票销毁社区预留剩余NAS总量的 \(\alpha\) \% , \(\alpha\) \% 是当前NAS 质押率占流通量的份额。

投票销毁通过细节:

\begin{enumerate}[a.]
  \item 投票需要满足 \(\alpha\) \%的现行NAX总量,才合格
  \item 支持销毁的比例超出50\%
  \item 投票销毁,只返还50\%
\end{enumerate}
