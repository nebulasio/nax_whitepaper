\section{NAX 应用场景展望}
正如上述,为了让整个系统设计简洁有效,NAX的核心设计逻辑里只有系统增发的简单逻辑,把其他的奖励,销毁逻辑交给了各个使用场景。根据我们的观察,每个使用场景,可以拥有自己特有的激励需求,不宜统一到系统逻辑当中。未来由社区沉淀下来的有效的激励手段,也有可能下沉到核心逻辑的可能。

\subsection{贡献度奖励和激励}
Go Nebulas平台,是星云进一步朝着Autonomous Metanet目标前进的一个重要一节。从星云基金会开始,逐步向社区发散,引入社区的力量,共同建设星云生态和愿景。星云基金会将会投入不少于300万NAS的资金,资助Go Nebulas平台上的项目。

\subsubsection{激励场景}
为了体现星云的核心理念,在Go Nebulas平台上,一切参与了星云社区建设的贡献者,除了获得项目回报之外,也将会得到相应的NAX的激励。此处NAX的激励来自于星云基金会投资的NAS资金池质押(Staking)所得NAX。具体激励细则将由Go Nebulas平台管理者来共同来决定。

在这里,我们可以对NAX在Go Nebulas平台的使用场景的展望:

提供更高的NAX激励给以下项目类别:
\begin{enumerate}
  \item 核心基础建设
  \item 市场拓展
  \item 推广拉新
\end{enumerate}

更多激励模式:
\begin{enumerate}
  \item 奖励过去在Go Nebulas做出杰出贡献的项目
  \item 邀请新人加入奖励,新人在系统中的贡献激励也会相应奖励给邀请人
\end{enumerate}

\subsubsection{消耗场景}
Go Nebulas平台除了是一个投入社区建设,获得NAX激励的重要方式之外,也同样是一个NAX的重要使用场景:
\begin{enumerate}
  \item 发起提案
  \item 申领项目
  \item 项目投票
\end{enumerate}

\subsection{链上治理相关}
当NAX变得很有效时,NAX也有可能成为星云治理的重要工具选择之一,为了鼓励大家社区治理的参与度,这些投票场景可能会增设一些由专项拨款的NAX作为投票激励。这些治理场景可能包括,基金会主席竞选,节点竞选等。

\subsubsection{节点竞选}
NAX的作为星云贡献激励,Staking的来源,一定程度上是社区贡献的一种凭证,是一个有效地结合主网PoD技术革新的一个有效的技术支持和手段。

\begin{enumerate}[a.]
  \item 从节点候选人由NAX投票选出来
  \item 参与节点竞选, 需要销毁相应量的NAX,并质押NAS
  \item 社区可以众筹成为节点,并拥有PoS Staking的收益
\end{enumerate}

\subsubsection{社区治理竞选}
\begin{enumerate}
  \item 参选需要筹集一定量的NAX以及质押NAS
  \item 可以使用NAX投票产生竞选结果
\end{enumerate}

\subsection{星云生态推广}
星云基金会扶持开发的,基础生态产品是星云链生态使用的重要入口,这里包含现在的,还有将来基金会孵化的星云产品,例如:NAS nano Pro \& Explorer \& Nebulas DEX等。随着Next DAO的推进以及社区治理的前进,社区里将会出现越来越多的NRC20 Token和治理尝试。这些币种都将需要相应的工具支持,所需要上NAS nano Pro和Explorer的需求。资源空间有限的情况下,为了使得星云生态相关的产品推出更多优秀好的项目,将可能会使用NAX作为平台的上币费用。这些收集到的上币费,也将投入到平台的建设当中。

\subsection{社区预留NAS销毁计划}
星云基金会曾提议销毁社区预留的3500w NAS,这个也将可能成为NAX的一个使用场景,具体是否销毁,销毁方式,如何销毁都会征求社区的意见。NAX只是会提供一个抛砖引玉的方案,例如:每个自然月1号发起一次使用NAX的投票销毁社区预留剩余NAS总量的 \(\alpha\) \% , \(\alpha\) \% 是当前NAS 质押率占流通量的份额。
