\section{Token Economy}

区块链技术带来的“去中心化”理念正在被用于越来越多的场景。作为区块链技术
的起源,比特币已经证实了去中心化对于数字资产的非凡意义;更进一步的,以太坊
证实了去中心化对于分布式应用的重要性;越来越多的区块链项目正在探索去中心化
这一理念在更多场景及应用下的价值。
不难发现,“去中心化”理念的背后,是区块链系统中的开放性(Openness)与匿
名性并存。
然而,区块链系统的这种特性在一定程度上造成了价值衡量体系的缺失[1]。这
反映在两方面。首先,由于区块链系统的匿名性,很难推断多个属于不同账户的数据
和资产是否属于同一个用户,这导致了区块链系统中不能构建类似HTTP Cookie [2]
的机制,也很难通过传统的数据分析技术从不同的角度分析用户特征;另一方面,区
块链系统的开放性又使得其面临着很强的操纵挑战,价值衡量体系很容易受到各种针
对性的操纵攻击,这不同于任何封闭的、独立的价值衡量体系。
我们认为有效的价值衡量体系是区块链生态能够繁荣发展的基础,价值衡量体
系的缺失或无效必然会限制整个区块链行业的发展。
首先,随着协作规模不断变大,并且对效率的要求不断升级,我们需要一个价值
衡量体系来为区块链系统以及区块链系统上的应用、数据和账户的价值提供可判断的
量化标准,否则要么因为无法量化评估而影响效率,要么因为评估失当引发不公平甚
至导致失控。
其次,区块链尚处于发展阶段,区块链上大量数据及资产的价值等着去发现。有
效的价值衡量体系将使得冰山下的部分得以露出,催生新兴应用甚至领域出现。比如
区块链上的借贷和征信类服务、数据搜索和个性化推荐、原生跨链交易和数据交换
等,价值衡量体系将使这些领域突破瓶颈。
最后,生态建设需要有效的激励和健康的发展方向。有效激励的基础就是有效的
价值衡量体系。如果没有价值衡量体系,甚至价值衡量体系是歪曲的,那么就会导致
激励机制失效,整个区块链系统不可避免地走向灭亡。
综上,一个区块链价值衡量标准需要具备三个特点:
• 真实性一个好的价值衡量标准应该能够准确反映出区块链经济系统的特征,这
样才能在相应的领域具有足够的公信力;
• 公平性价值衡量标准为相应的激励提供了依据,因此,这一依据必须足够公平,
1
才能防止作弊或操纵带来的“劣币驱逐良币”现象;
• 多样性需要使用数据及数字资产价值的场景可能是多种多样的,其使用方式及
对应的激励方式不尽相同,因此相应的价值衡量标准既不能脱离应用场景,亦
要满足前述的真实性及公平性。
星云指数(Nebulas Rank)将是一个满足以上三个特点的区块链价值衡量标准。
为了体现真实性,我们参考了诸多指标,最终我们定义星云指数为:衡量账户地
址对于区块链这一经济系统的贡献度。
本质上来说,区块链作为一个经济体,并不违背经典货币理论。我们认为区块链
系统之上的加密数字货币应该具备基本的货币属性,并且加密数字货币的价值源于其
流通性。因此,加密数字货币的交易记录是衡量加密数字货币这一经济体的有效数据
来源。更进一步的,我们认为每个账户发起的每一笔交易都在一定程度上增加了加密
数字货币的流通性。微观角度看,每个账户的交易行为都最终反映在了区块链系统的
价值中;从宏观角度看,我们将所有账户地址的星云指数的和定义为整个区块链系统
的经济总量。
为了验证星云指数设计的有效性,我们在基于以太坊的链上数据中计算了所有
账户的星云指数之和,并与Coinmarketcap.com 中同期的以太坊市值进行了对比。我
们的对比表明了二者具有很强的正相关性(0.84),即星云指数既能够在微观层面衡
量每个账户对经济系统的贡献,亦能在宏观层面反映整个经济系统总量的变化。
为了保证公平性,我们设计了能够有效抵抗操纵的计算函数。并论证出了星云指
数在抵抗操纵方面可达到的性能。
在星云指数的理论基础上,为了满足多样性的需要,我们将星云指数分为核心星
云指数(Core Nebulas Rank)和计算基于核心星云指数的扩展星云指数(Extended
Nebulas Ranks)两部分。
核心星云指数针对区块链中不同账户对于区块链系统的贡献度给出了计算方法。
其计算基于两个参考因素:其一,账户在一定时期内的资产中值;其二,账户在一定
时期内的出入度衡量。
扩展星云指数则基于核心星云指数来构建,针对区块链生态中各应用可能需要
的价值尺度给出了不同的计算方法,以便更符合不同场景的实际需要。并举了几种扩
展星云指数的计算方法作为参考,例如:如何根据核心星云指数对智能合约进行排
名;如何将星云指数拓展到多个维并给予不同的权重等。
本黄皮书除了给出理论论证,还解决了几个星云指数落地时必须面对的问题,例
如星云指数是否上链,星云指数的计算如何更新等。对星云指数实际落地给出了具体
的工作方向。

2008 年10 月,中本聪(Satoshi Nakamoto)公开发表比特币白皮书[4]。比特币
作为区块链的太初应用,践行了其作为“一个去中心化电子现金系统”的初衷。比特币
的产生不依赖于任何机构,而是根据特定算法,依靠大量计算产生,保证了比特币网
络分布式记账系统的一致性。
通过特定的脚本语言,我们可以利用比特币实现第三方支付交易、高效小额支付
(efficient micro-payments)等功能。此后涌现出许多以比特币为参照的试验品,在提
供基本的货币属性外,作出了更多的尝试。例如早期的域名币(Namecoin) [5] 提供
了一种去中心化的域名系统DNS。以及基于“货币染色(Colored coins)”的开放资产
项目(OpenAssets) [6],其本质都是模仿比特币,利用可追溯性,又复制了一份智能
资产。
很遗憾的是,比特币脚本语言的设计存在很多缺陷,如仅支持较少指令,且并不
具备图灵完备性,这使得其应用场景受限。
随着区块链技术研究的不断深入,涌现了更多后继者,尝试拓展和添加更多与应
用程序相关的功能。其中最令人瞩目的实现是以太坊(Ethereum) [7]。以太坊突破
性地提供了图灵完备的智能合约(Smart Contracts),从而大幅拓展了应用场景。
智能合约是区块链系统中可以用技术手段来强制执行的合约,以太坊智能合约
运行在以太坊虚拟机(Ethereum Virtual Machine)上,以太坊虚拟机不受任何实体
控制,通过共识算法来验证合约本身及其输出的完整性。
基于以太坊的智能合约,人们得以开发能实现复杂功能的分布式应用(DApp)。
3
除了基本交易功能外,DApp 为众多领域提供了解决方案,如投票、众筹、借贷、知
识产权等。
以太坊成功拓展了区块链的可能性,但以太坊缺少价值衡量标准,导致潜在杀手
级应用的落地和推广存在困难。
对于支持智能合约的区块链系统,其账户通常包括外部账户(Externally owned
account,EOA)和智能合约账户,对于这两类账户目前尚缺少合理的评价指标。同
时,在诸多交易以及智能合约的调用过程中,隐藏着难以估计的信息。后者相比传统
交易数据,往往具有更多维度,因此也无法使用传统价值衡量标准评估。
其实,早在2015 年,Chris Skinner 便提出了“价值网络(value web)”的理念[8],
其中提到,一个价值经济系统(Value ecosystem)应包括价值交换(value exchanges)、
价值存储(value stores)以及价值管理系统(value management systems)三部分,缺
一不可。同时Chris 也指出,对于比特币等数字加密货币来说,价值网络的衡量相比
传统社会价值有着明显不同,挑战更大。

根据费雪公式:
\begin{equation}
M * V = T / P
\end{equation}
\(M\)是Token数量,\(V\)是Token流通速度,\(P\)是Token价格,而\(T\)是系统内总交易额。很好理解,等式两边其实算的都是以Token数量为计量的GDP。左边\(M * V\)是个数乘以流通速度等于GDP(Token计量),右边总GDP(法币计量)除以Token价格(法币计量)也等于GDP(Token计量)。

通过这个公式,我们不难推出(以后补上推理步骤)通过增加持币价值是最终有效的提升币价和使用价值的方式。

\subsection{减少NAS流通量}
\subsubsection{增加质押}
\subsubsection{增加地址数}
\subsubsection{减少交易所存量}
\subsection{增加持有NAS和NAX的动力(减少交易动力)}
\subsection{增加NAX使用和消耗场景 (供需平衡)}
